\documentclass[12pt,a4paper]{report}
\usepackage[left=1.50cm, right=2.00cm, top=2.00cm, bottom=2.00cm]{geometry} % Tùy chỉnh lề
\usepackage{mathpazo} % Gói dùng để thay đổi phông chữ mặc định của tài liệu sang phông Palatino, và cung cấp phông chữ tương thích cho các ký hiệu toán học
\usepackage{graphicx} % Gói để chèn và xử lý hình ảnh: \includegraphics[cấu hình]{tên file} 
\usepackage[utf8]{vietnam} % Gói lệnh để gõ tiếng việt
\usepackage{amsmath, amssymb} % Gói lệnh toán học
\usepackage{amsfonts} % Gói dùng để mở rộng các ký hiệu toán học, đặc biệt là các ký hiệu tập hợp và phông chữ toán học đặc biệt
\usepackage{array} % Gói lệnh thêm các chức năng mở rộng cho bảng 
\usepackage{hhline} % Gói lệnh cải tiến hline: - một gạch ngang, = hai gạch ngang, ~ trống gạch ngang, # hai gạch xuống cắt qua hai gạch ngang, | một gạch xuống
\usepackage{xcolor} % Gói tùy chỉnh màu sắc: \textcolor{màu}{Văn bản}, \colorbox{màu}{Văn bản}, \fcolorbox{màu 1}{màu 2}{Văn bản}, \definecolor{tên màu mới}{mô hình màu sắc}{thông số màu}
\usepackage{listings} % Gói dùng để hiển thị mã nguồn (code) chuyên nghiệp và có tô màu cú pháp trong tài liệu
\usepackage{fancyhdr} % Gói dùng để tùy chỉnh header (đầu trang) và footer (chân trang) của tài liệu
\usepackage[hidelinks]{hyperref} % Gói tạo liên kết trong và ngoài tài liệu
\usepackage{tikz} % Gói để vẽ hình ảnh vector, sơ đồ, biểu đồ, hình học, hình minh họa, sơ đồ khối, cây phân cấp, đồ thị mạng,...
\usetikzlibrary{shapes.geometric, arrows} % Dùng để nạp thêm thư viện con cho TikZ, giúp vẽ các sơ đồ khối (flowchart) hoặc các hình học đặc biệt.
\definecolor{fitblue}{RGB}{0, 80, 150} % Mã màu gần giống với logo
\usepackage{tocloft}
\renewcommand{\cfttoctitlefont}{\LARGE\bfseries} % căn giữa và đổi kích thước
\setlength{\cftbeforetoctitleskip}{0pt}
\setlength{\cftaftertoctitleskip}{0pt}\usepackage{xcolor}     % Cho màu sắc
\usepackage{etoolbox}   % Cho phép chỉnh sửa môi trường LaTeX

\makeatletter
\renewcommand{\@cite}[2]{\textcolor{blue}{[#1\if@tempswa , #2\fi]}}
\makeatother

%------Gói lệnh hỗ trợ bảng------ 
\usepackage{multirow} % Gộp hàng
\usepackage{multicol} % Gộp cột
\usepackage{booktabs} % Tùy chỉnh bảng đẹp hơn
\usepackage{colortbl} % Thêm màu
\usepackage{longtable} % Kéo dài bảng qua các trang
\usepackage{fancybox}
\usepackage{setspace}
\usepackage{booktabs}
\usepackage{float}
%------Định nghĩa kiểu cột có khả năng tùy chỉnh kích thước-----
\newcolumntype{L}[1]{>{\raggedright\let\newline\\\arraybackslash\hspace{0pt}}m{#1}}
\newcolumntype{C}[1]{>{\centering\let\newline\\\arraybackslash\hspace{0pt}}m{#1}}
\newcolumntype{R}[1]{>{\raggedleft\let\newline\\\arraybackslash\hspace{0pt}}m{#1}}
%------Định nghĩa listings C++------
\lstset{
    frame=single,
    language=C++,
    aboveskip=3mm,
    belowskip=3mm,
    showstringspaces=false,
    columns=flexible,
    basicstyle={\normalsize\ttfamily},
    numbers=none, % Không hiển thị số dòng
    keywordstyle=\color{blue},
    commentstyle=\color{dkgreen},
    stringstyle=\color{mauve},
    breaklines=true,
    breakatwhitespace=true,
    tabsize=3
} 
\lstdefinestyle{numbered}{ % Style cho code có đánh số
    numbers=left, % Hiển thị số dòng bên trái
    numberstyle=\small\color{gray},
    numbersep=5pt, % Khoảng cách giữa số dòng và mã nguồn
    frame=single,
    framexleftmargin=15pt, % Đẩy khung bao quanh để chứa số dòng
    xleftmargin=15pt, % Lề trái để số dòng không bị cắt
    language=C++,
    aboveskip=3mm,
    belowskip=3mm,
    showstringspaces=false,
    columns=flexible,
    basicstyle={\normalsize\ttfamily},
    keywordstyle=\color{blue},
    commentstyle=\color{dkgreen},
    stringstyle=\color{mauve},
    breaklines=true,
    breakatwhitespace=true,
    tabsize=3
}
\begin{document}

\thispagestyle{empty}
\begin{center}
\thispagestyle{empty}  % Không đánh số
    \setlength{\fboxrule}{1.2pt}
    \doublebox{%
        \begin{minipage}{0.95\textwidth}
            \begin{center}
                \vspace{0.6cm}
                ĐẠI HỌC QUỐC GIA THÀNH PHỐ HỒ CHÍ MINH\\
                TRƯỜNG ĐẠI HỌC KHOA HỌC TỰ NHIÊN\\
                \textbf{KHOA CÔNG NGHỆ THÔNG TIN}\\[0.4em]
                -\hspace{0.01cm}-\hspace{0.01cm}-\hspace{0.01cm}-\hspace{0.01cm}-\hspace{0.01cm}-\textbf{oOo}-\hspace{0.01cm}-\hspace{0.01cm}-\hspace{0.01cm}-\hspace{0.01cm}-\hspace{0.01cm}-\\[2em]

                % Logo
                \includegraphics[width=0.20\textwidth]{logo-hcmus.png}\\[1em]

                \textbf{\LARGE BÁO CÁO BÀI TẬP LỚN MÔN HỌC}\\[0.5em]
                \textbf{CSC10004 - Cấu Trúc Dữ Liệu Và Giải Thuật}\\[1em]
                \textcolor{fitblue}{\LARGE{\textbf{NGHIÊN CỨU VÀ ĐÁNH GIÁ HIỆU NĂNG PHƯƠNG PHÁP FIBONACCI HASHING \\TRONG BẢNG BĂM}}}\\[5em]

                \begin{tabular}{ l  l }
                    \textbf{Giảng viên hướng dẫn}  &: CN. Lê Nhựt Nam \\
                    \textbf{Nhóm sinh viên thực hiện} &: Nhóm 9 -- Lớp 24CTT2A \\
                \end{tabular}\\[1.5em]

                % Bảng thành viên
                \renewcommand{\arraystretch}{1.2}
                \begin{tabular}{ c  l  c }
                
                    \textbf{STT} & \textbf{Họ và tên} & \textbf{MSSV} \\
                    
                    1 & Trương Tuấn Anh & 24120260 \\
                    2 & Tôn Thất Kiên & 24120078 \\
                    3 & Bùi Văn Thiên & 24120138 \\
                    4 & Bùi Trọng An & 24120252 \\
                
                \end{tabular}
                
                \vspace{11em}
                \textbf{TP. Hồ Chí Minh, tháng 6 năm 2025}
                \\[3em]
                \textbf{ }
            \end{center}
        \end{minipage}
    }
\end{center}

\newpage
\pagenumbering{arabic}
\setcounter{page}{1}

\chapter*{\centering{LỜI CAM KẾT}}
\addcontentsline{toc}{chapter}{\textcolor{fitblue}{LỜI CAM KẾT}}

\setlength{\parindent}{2em}   % Thụt đầu dòng mỗi đoạn
\setlength{\parskip}{0.5em}     % Dãn đoạn giữa các đoạn văn

\noindent \indent Nhóm chúng em xin cam kết rằng đồ án “\textit{Nghiên cứu và đánh giá hiệu năng phương pháp Fibonacci Hashing trong bảng băm}” là công trình do chính nhóm thực hiện dưới sự hướng dẫn của giảng viên hướng dẫn.

\textbf{\textit{Về tính độc lập và nguyên gốc:}}

- Toàn bộ nội dung nghiên cứu, phân tích và triển khai trong báo cáo đều do nhóm tự thực hiện, không sao chép từ bất kỳ nguồn nào một cách không hợp lệ.

- Các kết quả, thuật toán, biểu đồ, và đoạn mã nguồn (code implementation) đều được xây dựng và kiểm chứng trong quá trình học tập và nghiên cứu của các thành viên trong nhóm.

- Nhóm cam kết không sử dụng toàn bộ hoặc một phần nội dung từ các báo cáo, đồ án, tài liệu của người khác mà không có sự cho phép hoặc không trích dẫn rõ ràng.

\textbf{\textit{Về tài liệu tham khảo:}}

- Mọi tài liệu, dữ liệu và nguồn thông tin được sử dụng trong quá trình thực hiện đều được trích dẫn và ghi nguồn rõ ràng, đúng quy định học thuật.

- Các thuật toán, mô hình phân tích được tham khảo từ các tài liệu học thuật uy tín, đồng thời kết hợp với kiến thức tích lũy trong quá trình học tập tại trường.

- Tất cả ý tưởng, quan điểm không phải của nhóm đều được ghi nhận và dẫn nguồn cụ thể theo chuẩn học thuật.

\textbf{\textit{Về trách nhiệm:}}

- Nhóm hoàn toàn chịu trách nhiệm về tính chính xác và trung thực của toàn bộ nội dung báo cáo.

- Trong trường hợp có phát hiện sai sót, nhóm cam kết sẽ chủ động tiếp nhận và khắc phục kịp thời.

- Nhóm cam kết tuân thủ đầy đủ các quy định về đạo đức học thuật và nghiên cứu khoa học của khoa và nhà trường.

\begin{flushright}
TP. Hồ Chí Minh, tháng 6 năm 2025\\
Đại diện nhóm thực hiện \\
\end{flushright}

\chapter*{\centering{LỜI MỞ ĐẦU}}
\addcontentsline{toc}{chapter}{\textcolor{fitblue}{LỜI MỞ ĐẦU}}
\noindent \indent 

\vspace{2em}

\begin{flushright}
TP. Hồ Chí Minh, tháng 6 năm 2025\\
Nhóm sinh viên thực hiện
\end{flushright}




\renewcommand{\contentsname}{}        % Xóa "Mục lục"
\renewcommand{\listtablename}{}       % Xóa "Danh mục các bảng"
\renewcommand{\listfigurename}{}      % Xóa "Danh mục hình vẽ"

% MỤC LỤC
\chapter*{\centering{MỤC LỤC}}
\addcontentsline{toc}{chapter}{\textcolor{fitblue}{MỤC LỤC}}
\tableofcontents

% DANH MỤC BẢNG
\newpage
\chapter*{\centering{\MakeUppercase{Danh mục các bảng}}}
\vspace*{-7em}
\addcontentsline{toc}{chapter}{\textcolor{fitblue}{\MakeUppercase{Danh mục các bảng}}}
\listoftables

% DANH MỤC HÌNH VẼ
\newpage
\chapter*{\centering{\MakeUppercase{Danh mục các hình vẽ, đồ thị}}}
\addcontentsline{toc}{chapter}{\textcolor{fitblue}{\MakeUppercase{Danh mục các hình vẽ, đồ thị}}}
\listoffigures

\newpage
\chapter*{\centering{\MakeUppercase{Chương I. Giới thiệu}}}
\addcontentsline{toc}{chapter}{\textcolor{fitblue}{\MakeUppercase{Chương I. Giới thiệu}}}

\section*{1.1. Bối cảnh và lý do chọn đề tài}
\addcontentsline{toc}{section}{{1.1. Bối cảnh và lý do chọn đề tài}}
\subsection*{1.1.1. Bối cảnh}
\noindent \indent Trong thời đại công nghệ số, lượng dữ liệu được sinh ra mỗi ngày là khổng lồ, đi kèm với nhu cầu truy xuất, xử lý thông tin nhanh chóng và hiệu quả. Để giải quyết các bài toán này, bảng băm (\textit{hash table}) đã trở thành cấu trúc dữ liệu then chốt, được sử dụng trong hầu hết các hệ thống máy tính hiện đại.

Dẫn chứng 1: \textit{Hệ thống cơ sở dữ liệu và bộ nhớ đệm (\textit{cache})}.
Các hệ quản trị cơ sở dữ liệu như MySQL, MongoDB thường sử dụng bảng băm để lưu trữ các chỉ mục (\textit{index}), giúp tăng tốc truy vấn dữ liệu đến hàng triệu bản ghi. Trong các hệ thống web lớn như Facebook hay Google, bảng băm đóng vai trò trung tâm trong bộ nhớ đệm (Memcached, Redis), cho phép tìm kiếm dữ liệu được truy cập thường xuyên chỉ trong vài mili giây.

Dẫn chứng 2: \textit{Hệ thống từ điển, kiểm tra trùng lặp và mật mã}.
Hầu hết các phần mềm từ điển (cả offline và online) đều dùng bảng băm để tra cứu nghĩa từ gần như tức thì. Các thuật toán kiểm tra trùng lặp file (ví dụ: khi tải file lên Google Drive) cũng sử dụng hàm băm để so sánh nhanh hàng triệu file. Trong lĩnh vực an ninh mạng, các hàm băm mật mã (như SHA-256) giúp bảo mật dữ liệu, phát hiện sự thay đổi hoặc giả mạo thông tin.

Dẫn chứng 3: \textit{Thương mại điện tử và công nghệ lớn}.
Các sàn thương mại điện tử như Amazon, Shopee sử dụng hash table để quản lý tồn kho, đơn hàng, thông tin khách hàng. Các bộ máy tìm kiếm như Google Search sử dụng bảng băm trong việc xử lý index của hàng tỷ website, đảm bảo tốc độ tìm kiếm và độ tin cậy kết quả.

Dẫn chứng 4: \textit{Hệ thống phân tán và Blockchain}.
Công nghệ Blockchain cũng dựa vào các kỹ thuật băm để xác thực và liên kết các khối dữ liệu. Bảng băm phân tán (Distributed Hash Table – DHT) là thành phần cốt lõi trong các hệ thống mạng ngang hàng (P2P) như BitTorrent, giúp chia sẻ file hiệu quả trên hàng triệu máy tính.

\textbf{Tầm quan trọng của hàm băm tối ưu:}

Các dẫn chứng trên cho thấy: chỉ cần hàm băm không tốt, phân phối khóa không đều, số lần va chạm tăng cao, hiệu suất cả hệ thống sẽ bị ảnh hưởng rõ rệt. Ví dụ, chỉ cần bộ nhớ đệm của Google gặp hiện tượng clustering do khóa tuần tự, hàng triệu truy vấn mỗi giây có thể bị chậm lại, gây thất thoát doanh thu và trải nghiệm người dùng. Vì vậy, việc nghiên cứu các hàm băm tối ưu, trong đó có Fibonacci Hashing, không chỉ là vấn đề lý thuyết mà còn mang ý nghĩa thực tiễn rất lớn.

\subsection*{1.1.2. Lý do chọn đề tài}
\noindent \indent Như đã phân tích ở phần bối cảnh, bảng băm giữ vai trò sống còn trong rất nhiều hệ thống từ cơ sở dữ liệu, công nghệ web, thương mại điện tử đến các nền tảng blockchain, mạng xã hội và bảo mật thông tin. Tuy nhiên, thực tiễn triển khai cho thấy hiệu quả của bảng băm phụ thuộc chủ yếu vào chất lượng hàm băm. Nếu hàm băm phân phối khóa không đều, hệ thống sẽ phải đối mặt với hiện tượng va chạm, clustering, làm tăng thời gian truy xuất, giảm hiệu năng tổng thể – điều này có thể dẫn tới tổn thất lớn về tài nguyên, hiệu suất và thậm chí là trải nghiệm người dùng, như các dẫn chứng thực tiễn đã nêu.

Kỹ thuật \textit{modulo hashing} hiện nay vẫn phổ biến nhưng dễ gặp nhiều hạn chế, đặc biệt khi xử lý các tập khóa có tính quy luật hoặc lặp lại. Trong khi đó, \textit{Fibonacci Hashing }– dựa trên tỉ lệ vàng – được ghi nhận là có khả năng phân phối khóa đều hơn, giúp giảm số lần va chạm ngay cả với tập dữ liệu tuần tự hoặc cụm, tuy nhiên vẫn chưa được nghiên cứu và áp dụng rộng rãi trong thực tiễn tại Việt Nam.

Vì vậy, việc lựa chọn đề tài \textit{nghiên cứu và đánh giá hiệu năng phương pháp Fibonacci Hashing trong bảng băm} là hoàn toàn phù hợp, vừa mang tính thực tiễn, vừa góp phần bổ sung tri thức mới cho cộng đồng lập trình và khoa học dữ liệu trong nước. Đề tài không chỉ giúp nhóm sinh viên hiểu rõ bản chất các kỹ thuật băm mà còn có ý nghĩa tham khảo, định hướng ứng dụng cho các hệ thống lớn, nơi mà tối ưu hiệu năng luôn là vấn đề trọng tâm.



\section*{1.2. Lược sử về bảng băm, kỹ thuật modulo truyền thống và kỹ thuật Fibonacci Hashing}
\addcontentsline{toc}{section}{{1.2. Lược sử về bảng băm, kỹ thuật modulo truyền thống và kỹ thuật Fibonacci Hashing}}
\subsection*{1.2.1. Bảng băm – Từ ý tưởng đến ứng dụng rộng rãi}
\noindent \indent Bảng băm (\textit{hash table}) được giới thiệu lần đầu vào năm 1953 bởi Hans Peter Luhn, sau đó phát triển mạnh mẽ nhờ sự đóng góp của nhiều nhà khoa học máy tính như Donald Knuth. Ban đầu, bảng băm được xem như giải pháp tối ưu hóa thao tác tra cứu, thêm, xóa dữ liệu – đặc biệt trong các ứng dụng quản lý bộ nhớ, từ điển số và các thuật toán tìm kiếm. Trải qua hàng thập kỷ, bảng băm trở thành cấu trúc dữ liệu nền tảng, được tích hợp vào mọi ngôn ngữ lập trình hiện đại và các hệ thống dữ liệu lớn.

\subsection*{1.2.2. Kỹ thuật modulo hashing truyền thống}
\noindent \indent Từ những ngày đầu, hàm băm sử dụng phép chia lấy dư (\textit{modulo hashing}) là phương pháp phổ biến nhất nhờ tính đơn giản, dễ cài đặt, hiệu quả với các bộ dữ liệu nhỏ. Theo đó, một khóa (\textit{key}) được ánh xạ đến vị trí trong bảng băm thông qua công thức:

\[
\text{index} = \text{key} \% \text{tableSize}
\]

Nếu tableSize là lũy thừa của 2, phép modulo thậm chí có thể thay thế bằng toán tử AND bit để tăng tốc độ xử lý. Modulo hashing được sử dụng rộng rãi trong mọi hệ quản trị cơ sở dữ liệu, bộ nhớ đệm, và các cấu trúc như \texttt{unordered\_map} (C++), \texttt{HashMap} (Java), \texttt{dict} (Python)…

Tuy nhiên, modulo hashing dần bộc lộ những hạn chế nghiêm trọng khi bảng băm phải xử lý tập dữ liệu lớn, có quy luật, hoặc bị tấn công bởi các bộ khóa được thiết kế gây va chạm (\textit{collision}). Hiện tượng clustering, sự mất cân đối phân phối khóa khiến hiệu suất hệ thống giảm sút, thậm chí gây nghẽn cổ chai.

\subsection*{1.2.3. Kỹ thuật Fibonacci Hashing – Bước tiến về phân phối khóa}
\noindent \indent Nhằm khắc phục hạn chế của modulo hashing, các nhà toán học đã nghiên cứu những phương pháp mới tối ưu hóa khả năng phân phối khóa, trong đó nổi bật là Fibonacci Hashing (hay multiplicative hashing).

Fibonacci Hashing tận dụng đặc tính toán học của dãy Fibonacci và tỉ lệ vàng (golden ratio, ký hiệu $\varphi$ $\approx$ 1.618…). Bằng cách nhân khóa với một hằng số gần với tỉ lệ vàng rồi lấy phần nguyên tương ứng, hàm băm này tạo ra hiện tượng "quấn vòng" (\textit{wrapping}), giúp phân phối khóa đồng đều hơn khắp bảng, giảm tối đa khả năng clustering dù dữ liệu tuần tự hoặc lặp lại.

Công thức tổng quát cho bảng băm kích thước $2^n$:

\[
\text{index} = \left( \text{key} \times \varphi \right) \gg (\text{wordSize} - n)
\]

Với wordSize là số bit của kiểu dữ liệu (thường là 32 hoặc 64).

Kỹ thuật này được đề cập trong các tài liệu kinh điển như “The Art of Computer Programming” của Knuth, nhưng mãi đến gần đây mới được chú ý trở lại nhờ các nghiên cứu, blog chuyên sâu và ứng dụng trong các thư viện phần mềm hiệu suất cao (ví dụ: hash table trong ngôn ngữ Rust, một số project tối ưu của Google, Facebook…).

\section*{1.3. Khoảng trống nghiên cứu kỹ thuật Fibonacci Hashing}
\addcontentsline{toc}{section}{{1.3. Khoảng trống nghiên cứu kỹ thuật Fibonacci Hashing}}
\noindent \indent Dù \textit{Fibonacci Hashing} đã được đề xuất từ khá sớm trong các tài liệu kinh điển về cấu trúc dữ liệu và thuật toán, như “The Art of Computer Programming” của Donald Knuth, nhưng thực tế cho thấy số lượng công trình, tài liệu và dự án thực nghiệm về kỹ thuật này vẫn còn khá khiêm tốn, đặc biệt là ở môi trường học thuật và ứng dụng phần mềm tại Việt Nam.

Trong khi đó, các tài liệu giảng dạy cũng như các bài báo khoa học bằng tiếng Việt chủ yếu tập trung vào kỹ thuật băm modulo truyền thống hoặc các cải tiến phổ biến khác như double hashing, quadratic probing. Số liệu thực nghiệm, so sánh cụ thể về hiệu quả của Fibonacci Hashing trên các tập dữ liệu đặc thù như tập khóa tuần tự, tập khóa phân cụm hay tập khóa ngẫu nhiên, vẫn còn rất ít được đề cập hoặc chỉ dừng lại ở mức lý thuyết.

Ngoài ra, nhiều thư viện chuẩn trong các ngôn ngữ lập trình thông dụng (C++, Java, Python…) hiện nay cũng chưa tích hợp sẵn Fibonacci Hashing hoặc chưa tối ưu cho trường hợp sử dụng tỉ lệ vàng làm hệ số băm. Điều này dẫn tới việc các lập trình viên Việt Nam cũng như sinh viên công nghệ thông tin còn ít tiếp cận và thực hành đánh giá kỹ thuật này trong thực tiễn.

Bên cạnh đó, với xu hướng xử lý dữ liệu lớn, hệ thống phân tán, AI, IoT và điện toán đám mây ngày càng phổ biến tại Việt Nam, đòi hỏi ngày càng cao về cấu trúc dữ liệu có hiệu năng tối ưu, thì việc nghiên cứu, thử nghiệm và phân tích sâu hơn về Fibonacci Hashing là vô cùng cần thiết.

Tóm lại, vẫn còn một khoảng trống lớn trong việc kiểm chứng thực nghiệm, đánh giá khách quan và phổ biến kiến thức về Fibonacci Hashing trong cộng đồng khoa học dữ liệu và lập trình tại Việt Nam, đặc biệt là về các khía cạnh:

- Hiệu quả phân phối khóa so với modulo hashing truyền thống trên các kiểu dữ liệu khác nhau;

- Ảnh hưởng đến số lần va chạm, hiệu năng thao tác và khả năng mở rộng;

- Khả năng ứng dụng vào các hệ thống thực tiễn.

Chính vì vậy, việc chọn nghiên cứu và đánh giá kỹ thuật Fibonacci Hashing trong đề tài này không chỉ có ý nghĩa về mặt học thuật mà còn giúp mở rộng khả năng ứng dụng trong các bài toán thực tế hiện đại.

\section*{1.4. Mục tiêu và nhiệm vụ nghiên cứu}
\addcontentsline{toc}{section}{{1.4. Mục tiêu và nhiệm vụ nghiên cứu}}
\subsection*{1.4.1. Mục tiêu nghiên cứu}
\noindent \indent Mục tiêu tổng quát: Nghiên cứu, hiện thực và đánh giá hiệu quả của kỹ thuật Fibonacci Hashing trong xây dựng bảng băm, từ đó so sánh với kỹ thuật modulo hashing truyền thống nhằm chỉ ra ưu nhược điểm, điều kiện ứng dụng thực tế của Fibonacci Hashing.

Mục tiêu cụ thể:

- Hiểu rõ cơ sở lý thuyết, nguyên lý toán học và cách hiện thực của Fibonacci Hashing.

- Hiện thực thành công bảng băm sử dụng Fibonacci Hashing bằng ngôn ngữ C++.

- Thực hiện so sánh thực nghiệm về hiệu năng, số lần va chạm, khả năng phân phối khóa giữa Fibonacci Hashing và modulo hashing.

- Phân tích, rút ra kết luận, đề xuất khả năng áp dụng Fibonacci Hashing trong các hệ thống thực tế.
\subsection*{1.4.2. Nhiệm vụ nghiên cứu}
\noindent \indent Tìm hiểu, tổng hợp các tài liệu khoa học về bảng băm, modulo hashing và Fibonacci Hashing.

Xây dựng chương trình minh họa bảng băm sử dụng cả hai phương pháp: modulo hashing và Fibonacci Hashing.

Thiết kế bộ dữ liệu thử nghiệm đa dạng (random, tuần tự, cụm) để đánh giá khách quan hai kỹ thuật.

Đo lường, thu thập số liệu thực nghiệm: thời gian thực thi, số va chạm, hệ số tải, phân phối khóa…

Trực quan hóa kết quả bằng bảng số liệu, biểu đồ minh họa.

Phân tích, so sánh kết quả và tổng hợp báo cáo khoa học.

Đề xuất các tình huống thực tiễn, phạm vi ứng dụng hiệu quả của Fibonacci Hashing so với các kỹ thuật truyền thống.

\section*{1.5. Đối tượng và phạm vi nghiên cứu}
\addcontentsline{toc}{section}{{1.5. Đối tượng và phạm vi nghiên cứu}}
\subsection*{1.5.1. Đối tượng nghiên cứu}
\noindent \indent Các kỹ thuật băm (hashing) trong bảng băm, đặc biệt là hai kỹ thuật: modulo hashing (băm chia dư truyền thống) và Fibonacci Hashing (băm theo tỉ lệ vàng).

Các yếu tố ảnh hưởng đến hiệu suất bảng băm: số lần va chạm (collision), hiệu quả phân phối khóa, thời gian truy xuất, khả năng mở rộng.
\subsection*{1.5.2. Phạm vi nghiên cứu}
\noindent \indent Hiện thực và đánh giá các kỹ thuật băm trên bảng băm một chiều, không đi sâu vào các phương pháp xử lý va chạm nâng cao (double hashing, quadratic probing, rehash động...).

Sử dụng ngôn ngữ lập trình C++, tập trung vào hai kỹ thuật băm là modulo hashing và Fibonacci Hashing.

Thực nghiệm trên các tập dữ liệu kích thước vừa và nhỏ, bao gồm: dữ liệu ngẫu nhiên, dữ liệu tuần tự, dữ liệu phân cụm.

Đánh giá dựa trên các chỉ số: thời gian thao tác (insert, search, delete), số va chạm, hệ số tải (load factor), chiều dài cụm va chạm lớn nhất.

Không đi sâu vào các hệ thống bảng băm phân tán hoặc tích hợp bảo mật (hash cryptography).

\section*{1.6. Phương pháp nghiên cứu}
\addcontentsline{toc}{section}{{1.6. Phương pháp nghiên cứu}}

\noindent \indent Nghiên cứu tài liệu:
Tổng hợp lý thuyết từ sách, giáo trình, bài báo khoa học về bảng băm, modulo hashing, Fibonacci Hashing, phân tích ưu nhược điểm của từng phương pháp.

Thực nghiệm lập trình:
Xây dựng chương trình hiện thực hai bảng băm sử dụng modulo hashing và Fibonacci Hashing bằng ngôn ngữ C++. Tổ chức code rõ ràng, chú thích đầy đủ để tiện so sánh.

Thiết kế thí nghiệm:
Tạo các tập dữ liệu thử nghiệm đa dạng (ngẫu nhiên, tuần tự, phân cụm) để đánh giá khách quan hiệu năng hai phương pháp.

Đo lường và thu thập số liệu:
Ghi nhận các thông số trong quá trình thực nghiệm: thời gian thao tác (insert, search, delete), số va chạm, hệ số tải, chiều dài cụm va chạm.

Phân tích, trực quan hóa:
Tổng hợp số liệu, vẽ biểu đồ, so sánh kết quả giữa hai kỹ thuật. Đánh giá khách quan ưu nhược điểm và đề xuất khả năng ứng dụng.

    \noindent \indent

\section*{1.7. Cấu trúc và kế hoạch thực hiện báo cáo}
\addcontentsline{toc}{section}{{1.7. Cấu trúc và kế hoạch thực hiện báo cáo}}
\subsection*{1.7.1. Cấu trúc báo cáo}
\noindent \indent

\subsection*{1.7.2. Kế hoạch thực hiện}
\noindent \indent 

\newpage
\chapter*{\centering{\MakeUppercase{Chương II. Tổng quan về kỹ thuật băm và phương pháp Fibonacci Hashing}}}
\addcontentsline{toc}{chapter}{\textcolor{fitblue}{\MakeUppercase{Chương II. Tổng quan về kỹ thuật băm và phương pháp Fibonacci Hashing}}}

\section*{2.1. Tổng quan về bảng băm}
\addcontentsline{toc}{section}{{2.1. Tổng quan về bảng băm}}
\noindent \indent Bảng băm (hash table) là một cấu trúc dữ liệu hiệu quả, cho phép lưu trữ và truy xuất các cặp khóa – giá trị (key–value) với thời gian trung bình gần như hằng số, tức là \texttt{O(1)} \cite{cormen2009}. Nguyên lý hoạt động của bảng băm dựa trên một hàm băm (hash function) có nhiệm vụ ánh xạ khóa đầu vào \texttt{k} tới một vị trí chỉ số trong bảng có kích thước \texttt{m}, thông qua công thức \texttt{h(k) = k mod m} hoặc các biến thể phù hợp với kiểu dữ liệu.

Cấu trúc bảng băm được ưa chuộng trong nhiều ứng dụng thực tiễn như từ điển ánh xạ, hệ thống tra cứu, kiểm tra trùng lặp, bộ nhớ đệm, và tối ưu thuật toán quy hoạch động. Tính ưu việt của bảng băm đến từ khả năng truy xuất trực tiếp, độc lập với kích thước dữ liệu, miễn là hàm băm được thiết kế tốt và phân phối đồng đều.

Tuy nhiên, một trong những thách thức cốt lõi của bảng băm là hiện tượng va chạm (collision), xảy ra khi hai khóa khác nhau cùng được ánh xạ tới một vị trí trong bảng. Hiện tượng này là không thể tránh khỏi, đặc biệt khi không gian khóa lớn hơn không gian bảng lưu trữ.
\begin{quote}
“Hash tables are among the most efficient data structures for fast access when the number of keys is large and key distribution is unpredictable.”
(\textit{Cormen, Leiserson, Rivest, \& Stein, 2009, p. 257}) \\
(Tạm dịch: Bảng băm là một trong những cấu trúc dữ liệu hiệu quả nhất để truy xuất nhanh khi số lượng khóa lớn và phân phối khóa không thể đoán trước được.)
\end{quote}
\section*{2.2. Vấn đề va chạm trong bảng băm}
\addcontentsline{toc}{section}{{2.2. Vấn đề va chạm trong bảng băm}}
\noindent \indent Va chạm là tình huống xảy ra khi \texttt{h($k_1$) = h($k_2$)} trong khi \texttt{$k_1 \neq k_2$}. Khi đó, bảng băm không thể chèn phần tử mới vào vị trí được chỉ định mà không có một cơ chế xử lý phù hợp. Nếu không được giải quyết hiệu quả, va chạm có thể khiến hiệu năng suy giảm nghiêm trọng, thậm chí khiến cấu trúc bảng băm trở nên tương đương với tìm kiếm tuyến tính (linear search).

Có hai nhóm chiến lược phổ biến để xử lý va chạm:

- \textit{Phương pháp liên kết ngoài (Chaining)}:
Mỗi ô trong bảng chứa một danh sách liên kết (hoặc cấu trúc danh sách động khác), nơi lưu trữ tất cả các phần tử cùng có cùng chỉ số băm. Phương pháp này đơn giản, dễ hiện thực, và hoạt động tốt khi bảng có kích thước nhỏ hơn nhiều so với số lượng khóa. Tuy nhiên, nó tiêu tốn thêm bộ nhớ và có thể dẫn đến độ phức tạp tăng dần về thời gian tìm kiếm.

- \textit{Phương pháp địa chỉ mở (Open Addressing)}:
Thay vì dùng danh sách, phương pháp này tìm một vị trí khác trong bảng để chèn phần tử, theo một chuỗi dò (probe sequence). Tất cả dữ liệu được lưu ngay trong bảng, tiết kiệm không gian, nhưng đòi hỏi hàm dò được thiết kế khéo léo để tránh hiện tượng kết cụm (clustering).

Các kỹ thuật dò thông dụng bao gồm:

- \textit{Linear probing}: Dò lần lượt từng ô tiếp theo. Dễ hiện thực nhưng dễ gây kết cụm chính (primary clustering).

- \textit{Quadratic probing}: Tăng khoảng cách dò theo bình phương, giúp giảm kết cụm một phần nhưng vẫn gặp hiện tượng secondary clustering.

- \textit{Double hashing}: Sử dụng một hàm băm thứ hai để xác định bước nhảy, cho chuỗi dò phân tán hơn, ít kết cụm và hiệu quả hơn khi load factor cao.
\begin{quote}
“Without an effective collision resolution strategy, hash tables may degrade into inefficient linear search.”
(\textit{Knuth, D. E., 1998, p. 545}) \\
(Tạm dịch: Nếu không có một chiến lược xử lý va chạm hiệu quả, bảng băm có thể thoái hóa thành tìm kiếm tuyến tính kém hiệu quả.)
\end{quote}
\noindent \indent Khả năng xử lý va chạm tốt là yếu tố quyết định để bảng băm giữ được hiệu suất cao trong môi trường thực tế. Trong số các kỹ thuật địa chỉ mở, Fibonacci được xem là một trong những chiến lược ưu việt nhất hiện nay.

\section*{2.3. Kỹ thuật Fibonacci Hashing}
\addcontentsline{toc}{section}{{2.3. Kỹ thuật Fibonacci Hashing}}
\noindent \indent


\subsection*{2.3.4. Ứng dụng và nghiên cứu liên quan}
\noindent \indent 

\newpage
\chapter*{\centering{\MakeUppercase{Chương III. Cài đặt và đánh giá thực nghiệm}}}
\addcontentsline{toc}{chapter}{\textcolor{fitblue}{\MakeUppercase{Chương III. Cài đặt và đánh giá thực nghiệm}}}

\newpage
\chapter*{\centering{\MakeUppercase{Chương IV. Phân tích và đánh giá kết quả}}}
\addcontentsline{toc}{chapter}{\textcolor{fitblue}{\MakeUppercase{Chương IV. Phân tích và đánh giá kết quả}}}

\newpage
\chapter*{\centering{\MakeUppercase{Chương V. Kết luận và hướng phát triển}}}
\addcontentsline{toc}{chapter}{\textcolor{fitblue}{\MakeUppercase{Chương V. Kết luận và hướng phát triển}}}

\section*{5.1. Những gì đã làm được}
\addcontentsline{toc}{section}{{5.1. Những gì đã làm được}}

\section*{5.2. Hạn chế}
\addcontentsline{toc}{section}{{5.2. Hạn chế}}

\section*{5.3. Đề xuất cải tiến, mở rộng}
\addcontentsline{toc}{section}{{5.3. Đề xuất cải tiến, mở rộng}}

\newpage
\phantomsection
\addcontentsline{toc}{chapter}{\textcolor{fitblue}{\MakeUppercase{Tài liệu tham khảo}}}
\begin{thebibliography}{9}

\bibitem{cormen2009}
Cormen, T. H., Leiserson, C. E., Rivest, R. L., \& Stein, C. (2009). \textit{Introduction to Algorithms} (3rd ed.). MIT Press.

\bibitem{knuth1998}
Knuth, D. E. (1998). \textit{The Art of Computer Programming}, Vol. 3: \textit{Sorting and Searching} (2nd ed.). Addison-Wesley.

\bibitem{weiss2012}
Weiss, M. A. (2012). \textit{Data Structures and Algorithm Analysis in C++} (4th ed.). Pearson.

\bibitem{khan2019}
Khan, M. A., Qureshi, M. A., \& Iqbal, T. (2019). Evaluation of Hashing Techniques for High Load Environments. \textit{International Journal of Computer Applications}, 180(32), 1–6.

\bibitem{jiang2021}
Jiang, H., Wang, L., \& Zhang, T. (2021). Optimized Hashing in Compiler Memory Systems. ACM Transactions on Architecture and Code Optimization, 18(2), 1–25.

\bibitem{luhn1953}
Luhn, H. P. (1953). A system for automatic keyword assignment. \textit{IBM Journal of Research and Development}.

\bibitem{carter1979}
Carter, L., \& Wegman, M. N. (1979). Universal classes of hash functions. \textit{Journal of Computer and System Sciences}, 18(2), 143–154.

\bibitem{brent1973}
Brent, R. P. (1973). Reducing the retrieval time of scatter storage techniques. \textit{Communications of the ACM}, 16(2), 105–109.

\end{thebibliography}


\end{document}
